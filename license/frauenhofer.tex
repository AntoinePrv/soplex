\documentclass[11pt]{article}
\usepackage{german}
\usepackage{a4}
\usepackage{palatino}
\usepackage[latin1]{inputenc}
\parindent0ex
\pagestyle{empty}
\begin{document}
{\large
Lizenzvertrag zwischen dem
\begin{center}
Konrad-Zuse-Zentrum f�r Informationstechnik Berlin\\
Takustra�e 7, 14195 Berlin-Dahlem\\
\end{center}
im folgenden ZIB genannt und der
\begin{center}
Fraunhofer Gesellschaft f�r angewandte Forschung e.V.\\
f�r ihr\\
Fraunhofer-Institute for Algorithms
and Scientific Computing\\
Schloss Birlinghoven, 53754 Sankt Augustin
\end{center}
im folgenden Lizenznehmer genannt, 
�ber das Softwarepaket SoPlex Version 1.2.1, im Quellcode
einschlie�lich der dazugeh�rigen Dokumentation, 
im folgenden Software genant:
}

\begin{enumerate}

\item Das ZIB erteilt dem Lizenznehmer eine zeitlich, r�umlich und
   inhaltlich unbeschr�nkte, nicht ausschlie�liche Lizenz zur
   Benutzung und Verwertung des Softwarepaketes SoPlex.

   Der Lizenznehmer ist unter Beachtung der nachstehenden Bedingungen
   berechtigt, sein Nutzungsrechte auch im Wege der Unterlizenz auszu�ben.

   Der Lizenznehmer ist insbesondere berechtigt, die Software f�r
   eigene Zwecke zu nutzen, weiter zu entwickeln, in andere
   Softwaresysteme zu integrieren und die Software zu diesem Zweck zu
   vervielf�ltigen, zu bearbeiten, zu ver�ndern, vorzuf�hren und zu
   �bermitteln.

   Insbesondere darf die Software als sogennante Bibliothek in
   bin�rer Form an Softwareprodukte gebunden und eingesetzt werden.
   In dieser Form unterliegt die Weitergabe der Software keinen
   Beschr�nkungen. 

   Eine Weitergabe der Software im Quellcode an Dritte, 
   sofern diese an Projektpartner des Lizenznehmers im
   Rahmen eines Forschungsprojekts zur ausschlie�lichen Verwendung im
   und w�hrend des Projekts erfolgt, ist zul�ssig, aber dem ZIB
   anzuzeigen. 

   In allen �brigen F�llen einer Weitergabe bzw. (Unter-)Lizenzierung
   der Software als Quellcode an Dritte bedarf diese der vorherigen
   schriftlichen Zustimmung des ZIB.

\item Das ZIB gew�hrleistet, die Software selbst entwickelt zu haben und da�
     die nach diesem Vertrag einger�umten Nutzungsrechte des Lizenznehmers
     keine Rechte Dritter verletzen.

     Alle hier�ber hinausgehenden Gew�hrleistungsanspr�che an das 
     ZIB sind ausgeschlossen.

\item Diese Lizenz gilt auch f�r alle nachfolgenden Versionen der
   Software, die vom ZIB frei gegeben werden.

\item Bei Vertragsabschlu� gilt die Software als vollst�ndig �bergeben.

\item Das Eigentumsrecht an der Software verbleibt beim ZIB.

\item Das Urheberrecht f�r die Software verbleibt beim ZIB. 

\item Das ZIB �berl��t dem Lizenznehmer die Software \glqq so, wie sie
   ist\grqq und steht nicht daf�r ein, da� die Software f�r die
   Zwecke des Lizenznehmers brauchbar ist.

\item Das ZIB �bernimmt keine Wartungsverpflichtungen.

\item Das ZIB �bernimmt keine Haftung f�r Sch�den jeglicher Art, die
   sich aus der Verwendung der Software ergeben. 

\item Lizenznehmer zahlt f�r die Lizenz nach Rechnungstellung durch
   das ZIB eine einmalige Lizenzgeb�hr in H�he von Euro 18.000
   zuz�glich der gesetzlichen Mehrwertsteuer.
 
\item Als Gerichtsstand ist Berlin vereinbart.

\end{enumerate}

\bigskip
\begin{minipage}{4cm}
Berlin, den 18.11.2002

\vskip2cm
Henry Thieme\\                        
Verwaltungsleiter ZIB
\end{minipage}
\hfill
\begin{minipage}{5cm}
St. Augustin, den

\vskip2cm
Dr. Ralf Heckmann\\

\end{minipage}
\end{document}

