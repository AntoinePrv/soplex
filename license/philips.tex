\documentclass[11pt]{article}
\usepackage{german}
\usepackage{a4}
\usepackage{palatino}
\usepackage[latin1]{inputenc}
\parindent0ex
\pagestyle{empty}
\begin{document}
{\large
Lizenzvertrag zwischen dem
\begin{center}
Konrad-Zuse-Zentrum f�r Informationstechnik Berlin\\
Takustra�e 7, 14195 Berlin-Dahlem\\
-- im folgenden ZIB genannt --
\end{center}
und der
\begin{center}
Philips Medical Systems\\
X-Ray Predevelopment\\
Veenpluis 4-6,
5680 DA Best, The Netherlands\\
-- im folgenden Lizenznehmer genannt --
\end{center}
�ber das Softwarepaket SoPlex Version 1.2.1, im Quellcode
einschlie�lich der dazugeh�rigen Dokumentation\\
-- im folgenden Software genannt --
}

\begin{enumerate}

\item Das ZIB erteilt dem Lizenznehmer eine nicht-ausschlie�liche, zeitlich,
   r�umlich und inhaltlich unbeschr�nkte, unwiderrufliche und nicht
   �ber\-trag\-bare 
   Lizenz zur Benutzung des Softwarepaketes SoPlex zu nachstehenden
   Bedingungen:

   Die Software darf ausschlie�lich zu internen Zwecken des
   Lizenznehmers beim Lizenznehmer eingesetzt werden.

   Die Lizenz erlaubt maximal die gleichzeitige Installation und Verwendung 
   von zehn (10) Kopien der Software beim Lizenznehmer.

   Eine Weitergabe der Software an Dritte, ist in jeglicher
   Form unzul�ssig. Die Weitergabe von Lizenzen oder die 
   Vergabe von Unterlizenzen ist ebenfalls unzul�ssig. 
   
\item Das ZIB gew�hrleistet, die Software selbst entwickelt zu haben und dass
     die nach diesem Vertrag einger�umten Nutzungsrechte des Lizenznehmers
     keine Rechte Dritter verletzen.

     Alle hier�ber hinausgehenden Gew�hrleistungsanspr�che an das 
     ZIB sind ausgeschlossen.

\item Diese Lizenz gilt auch f�r alle nachfolgenden Versionen der
   Software, die vom ZIB frei gegeben werden.

\item Bei Vertragsabschlu� gilt die Software als vollst�ndig �bergeben.

\item Der Lizenznehmer zahlt f�r die Lizenz nach Rechnungstellung durch
   das ZIB eine einmalige Lizenzgeb�hr in H�he von Euro 1.000,00
   zu\-z�g\-lich der gesetzlichen Mehrwertsteuer.

\item Das Eigentumsrecht an der Software verbleibt beim ZIB.

\item Das Urheberrecht f�r die Software verbleibt beim ZIB. 

\item Das ZIB �berl��t dem Lizenznehmer die Software \glqq so, wie sie
   ist\grqq und steht nicht daf�r ein, dass die Software f�r die
   Zwecke des Lizenznehmers brauchbar ist.

\item Das ZIB �bernimmt keine Wartungsverpflichtungen.

\item Das ZIB �bernimmt keine Haftung f�r Sch�den jeglicher Art, die
   sich aus der Verwendung der Software ergeben. 
   Dies gilt nicht f�r Vorsatz oder
   grobe Fahrl�ssigkeit. F�r diesen Fall wird die H�he der Haftungssumme
   beschr�nkt auf den Betrag gem�� Ziffer 5.

\item Sollten einzelne Bestimmungen dieser Vereinbarung unwirksam sein
   oder werden, so wird dadurch die G�ltigkeit der �brigen Bestimmungen
   nicht ber�hrt. Die unwirksamen Bestimmungen sind in diesem Fall durch
   Regelungen zu ersetzen, die dem wirtschaftlichen und rechtlichen Zweck
   der unwirksamen Bestimmungen am n�chsten kommen.

\item Als Gerichtsstand ist Berlin vereinbart.

\end{enumerate}

\bigskip
\begin{minipage}{4cm}
Berlin, den 16.6.2004

\vskip2cm
Henry Thieme\\                        
Verwaltungsleiter ZIB
\end{minipage}
\hfill
\begin{minipage}{6cm}
Best, den

\vskip2cm
Hein Haas\\
Director X-Ray Predevelopment
\end{minipage}
\end{document}



